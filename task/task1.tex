\documentclass[12pt
a4paper,
parskip=full ]{article}

\usepackage[utf8]{inputenc} 
\usepackage[T1]{fontenc} 
\usepackage{graphicx}
\usepackage{svg}

\title{\textbf{Werkzeuge fuer das wissenschaftliche Arbeiten}\\ Phyton for Machine Learning and Data Science}
\author{}
\date{Abgabe: 15.12.2023}

\begin{document}

\maketitle
\hrule
\vspace{1cm}

\tableofcontents
\vspace{1cm}
\hrule

\section{Projektaufgabe}
In dieser Aufgabe besch\"aftigen wir uns mit der Objektorientierung in Python. Der Fokus liegt auf der Implementierung einer Klasse, dabei nutzen wir insbesondere auch Magic Methods.

\begin{figure}[h!]
    \centering
    \includesvg[width=0.9\textwidth]{../diagram/classes_files}
    \caption{Darstellung der Klassenbeziehungen.}
    \label{fig:classes}
\end{figure}



\subsection{Einleitung}
Ein Datensatz besteht aus mehreren Daten; ein einzelnes Datum wird durch ein Objekt der Klasse \texttt{DataSetItem} repr\"asentiert. Jedes Datum hat einen Namen (Zeichenkette), eine ID (Zahl) und beliebigen Inhalt.

Nun sollen mehrere Daten, also Objekte vom Typ \texttt{DataSetItem}, in einem Datensatz zusammengefasst werden. Sie haben sich schon auf eine Schnittstelle und die ben\"otigten Operationen geeinigt, die ein Datensatz unterst\"utzen muss. Es gibt eine Klasse \texttt{DataSetInterface}, die die Schnittstelle definiert und die Operationen jedes Datensatzes angibt. Bisher fehlt jedoch noch die Implementierung eines Datensatzes mit allen Operationen.

Implementieren Sie eine Klasse \texttt{DataSet} als Unterklasse von \texttt{DataSetInterface}.

\subsection{Aufbau}
Es gibt drei Dateien:
\begin{itemize}
    \item \texttt{dataset.py}: Enth\"alt die Klassen \texttt{DataSetInterface} und \texttt{DataSetItem}.
    \item \texttt{implementation.py}: Hier muss die Klasse \texttt{DataSet} implementiert werden.
    \item \texttt{main.py}: Nutzt die Klassen \texttt{DataSet} und \texttt{DataSetItem} aus den jeweiligen Dateien und testet die Schnittstelle und die Operationen von \texttt{DataSetInterface}.
\end{itemize}


\subsection{Methoden}
Die Klasse \texttt{DataSet} muss insbesondere die folgenden Methoden implementieren. Die genaue Spezifikation finden Sie in der Datei \texttt{dataset.py}:

\begin{itemize}
    \item \texttt{\_\_setitem\_\_(self, name, id\_content)}: Hinzuf\"ugen eines Datums mit Name, ID und Inhalt.
    \item \texttt{\_\_iadd\_\_(self, item)}: Hinzuf\"ugen eines \texttt{DataSetItem}.
    \item \texttt{\_\_delitem\_\_(self, name)}: L\"oschen eines Datums anhand des Namens. Der Name eines Datums ist ein eindeutiger Schl\"ussel und darf nur einmal pro Datensatz vorkommen.
    \item \texttt{\_\_contains\_\_(self, name)}: Pr\"ufung, ob ein Datum mit diesem Namen im Datensatz vorhanden ist.
    \item \texttt{\_\_getitem\_\_(self, name)}: Abrufen des Datums \"uber seinen Namen.
    \item \texttt{\_\_and\_\_(self, dataset)}: Bestimmung der Schnittmenge zweier Datens\"atze und R\"uckgabe als neuen Datensatz.
    \item \texttt{\_\_or\_\_(self, dataset)}: Bestimmung der Vereinigung zweier Datens\"atze und R\"uckgabe als neuen Datensatz.
    \item \texttt{\_\_iter\_\_(self)}: Iteration \"uber alle Daten des Datensatzes (optional mit einer Sortierung).
    \item \texttt{filtered\_iterate(self, filter)}: Gefilterte Iteration \"uber einen Datensatz. Eine Lambda-Funktion mit den Parametern Name und ID dient als Filter.
    \item \texttt{\_\_len\_\_(self)}: Abrufen der Anzahl der Daten in einem Datensatz.
\end{itemize}

\section{Abgabe}
Programmieren Sie die Klasse \texttt{DataSet} in der Datei \texttt{implementation.py}, um die oben beschriebene Aufgabe im VPL zu l\"osen. Alternativ k\"onnen Sie die Aufgabe auch direkt auf Ihrem Computer bearbeiten. Dazu finden Sie alle drei ben\"otigten Dateien im Moodle zum Download.

Das VPL nutzt den gleichen Code, wobei die Datei \texttt{main.py} um weitere Testf\"alle und \"Uberpr\"ufungen erweitert wurde. Diese \"Uberpr\"ufungen dienen dazu, sicherzustellen, dass Sie die richtigen Klassen verwenden.

\vspace{0.5cm}
\hrule
\vspace{0.2cm}
\noindent \textbf{Hinweis:} Die Dateien befinden sich im Ordner \texttt{/code/} dieses Git-Repositories.

\end{document}